\documentclass[]{elsarticle} %review=doublespace preprint=single 5p=2 column
%%% Begin My package additions %%%%%%%%%%%%%%%%%%%
\usepackage[hyphens]{url}

  \journal{Astronomy \& Astrophysics} % Sets Journal name


\usepackage{lineno} % add
\providecommand{\tightlist}{%
  \setlength{\itemsep}{0pt}\setlength{\parskip}{0pt}}

\bibliographystyle{elsarticle-harv}
\biboptions{sort&compress} % For natbib
\usepackage{graphicx}
\usepackage{booktabs} % book-quality tables
%%%%%%%%%%%%%%%% end my additions to header

\usepackage[T1]{fontenc}
\usepackage{lmodern}
\usepackage{amssymb,amsmath}
\usepackage{ifxetex,ifluatex}
\usepackage{fixltx2e} % provides \textsubscript
% use upquote if available, for straight quotes in verbatim environments
\IfFileExists{upquote.sty}{\usepackage{upquote}}{}
\ifnum 0\ifxetex 1\fi\ifluatex 1\fi=0 % if pdftex
  \usepackage[utf8]{inputenc}
\else % if luatex or xelatex
  \usepackage{fontspec}
  \ifxetex
    \usepackage{xltxtra,xunicode}
  \fi
  \defaultfontfeatures{Mapping=tex-text,Scale=MatchLowercase}
  \newcommand{\euro}{€}
\fi
% use microtype if available
\IfFileExists{microtype.sty}{\usepackage{microtype}}{}
\usepackage{longtable}
\usepackage{graphicx}
%% We will generate all images so they have a width \maxwidth. This means
%% that they will get their normal width if they fit onto the page, but
%% are scaled down if they would overflow the margins.
%\makeatletter
%\def\maxwidth{\ifdim\Gin@nat@width>\linewidth\linewidth
%\else\Gin@nat@width\fi}
%\makeatother
%\let\Oldincludegraphics\includegraphics
%\renewcommand{\includegraphics}[1]{\Oldincludegraphics[width=\maxwidth]{#1}}
\ifxetex
  \usepackage[setpagesize=false, % page size defined by xetex
              unicode=false, % unicode breaks when used with xetex
              xetex]{hyperref}
\else
  \usepackage[unicode=true]{hyperref}
\fi
\hypersetup{breaklinks=true,
            bookmarks=true,
            pdfauthor={},
            pdftitle={The Locus Algorithm},
            colorlinks=false,
            urlcolor=blue,
            linkcolor=magenta,
            pdfborder={0 0 0}}
\urlstyle{same}  % don't use monospace font for urls

\usepackage{xcolor}
\usepackage{graphicx}

\usepackage{gensymb}
%\setcounter{secnumdepth}{0}
%% Pandoc toggle for numbering sections (defaults to be off)
%\setcounter{secnumdepth}{0}
%% Pandoc header

\usepackage{caption}
\DeclareCaptionType{equ}[Expression][]

\begin{document}
\begin{frontmatter}

\title{The Locus Algorithm}


\author[DIAS,TUD]{Ois\'in Creaner\corref{cor1}}
\cortext[cor1]{Corresponding author}
\ead{creanero@cp.dias.ie}

\author[TUD]{Eugene Hickey}
\ead{Eugene.Hickey@it-tallaght.ie}
\author[TUD]{Kevin Nolan}
\ead{Kevin.Nolan@it-tallaght.ie}
\author[CIT]{Niall Smith}
\ead{nsmith@cit.ie}

\address[DIAS]{Dublin Institute for Advanced Studies, 31 Fitzwilliam Place, Dublin 2, Ireland}
\address[TUD]{Technological University Dublin, Tallaght Campus, Dublin 24, Ireland}
\address[CIT]{Cork Institute of Technology, Bishopstown, Cork, Ireland}

  
  \begin{abstract}
  This Paper describes the design, implementation and operation of a new algorithm,
  The Locus Algorithm; which enables optimised differential photometry.
  For a given target, The Locus Algorithm identifies the pointing for
  which the resultant FoV includes the target and the maximum number of
  similar reference stars available, thus enabling optimised differential
  photometry of the target. The application of The Locus
  Algorithm to a target from the Sloan Digital Sky Survey to provide
  optimum differential photometry for that target is also described. The algorithm was also
  used to generate catalogues of pointings to optimise Quasars
  variability studies and to generate catalogues of optimised pointings in
  the search for Exoplanets via the transit method.
  \end{abstract}
  
 \end{frontmatter}

\hypertarget{introduction}{%
\section{Introduction}\label{introduction}}

Photometric variability studies involve identifying variations in
brightness of a celestial point source over time. Such studies are
hampered by the Earth's atmosphere, which causes first order and second
order extinction \citep{milone2011high,young1991precise}. Differential
Photometry mitigates the effect of the Earth's atmosphere by comparing
the brightness of a target to reference stars in the same Field of View
(FoV). Differential photometry can be optimised for the target by
choosing a pointing whose Field of View (FoV) includes the target and
the maximum number of reference stars of similar magnitude and colour.
\citep{milone2011high,young1991precise,howell2006handbook,honeycutt1992ccd}).

The Locus Algorithm enables optimised differential photometry by
identifying the pointing for which the resultant FoV includes the target
and the best set of similar reference stars available.

\hypertarget{conceptual-basis-to-the-locus-algorithm}{%
\section{Conceptual basis to The Locus
Algorithm}\label{conceptual-basis-to-the-locus-algorithm}}

A locus can be defined around any star such that a FoV centred on any
point on the locus will include the star at the edge of the FoV. For
fields containing stars close to one another, if one locus intersects
with another, they produce Points of Intersection (PoIs) as shown in Figure \ref{loci_concept}.

\begin{figure}
\centering
\includegraphics[width=\textwidth]{fig1.png}
\caption{\label{loci_concept}Diagrammatic representation of two stars with loci
(red and blue perimeter lines), which intersect and produce two Points
of Intersection (PoI's) circled in yellow.  Copied from \citet{creaner2016thesis}}
\end{figure}

A FoV centred on any such PoI will include both stars associated with
creating it. At Points of Intersection the set of stars that can be
included in a FoV changes.

The Locus Algorithm considers candidate reference stars in what is
termed a Candidate Zone (CZ) - the zone of sky centred on the target
within which a FoV can be selected which includes both the reference
star and the target. For Candidate Reference Stars within the CZ, 
loci are determined, and all relevant PoI are identified. Each PoI 
is assigned a score derived from the number and similarity of reference
stars included in a FoV centred on that PoI. The PoI with the highest score 
becomes the pointing for the target.

\hypertarget{locus-algorithm-design}{%
\section{Locus Algorithm Design}\label{locus-algorithm-design}}
Based on the conceptual outline above, this section provides a mathematical
definition of the Locus Algorithm and an explanation of the terms used in it.
Section \ref{example-implementation-of-the-locus-algorithm} below describes a
 worked example of this algorithm applied to a sample star, SDSS ID 1237680117417115655.
\hypertarget{definition-of-coordinate-system-and-locus}{%
\subsection{Definition of Coordinate System and
Locus}\label{definition-of-coordinate-system-and-locus}}

For computational efficiency, The Locus Algorithm considers a Field of
View to be a rectangular area on the sky orientated such that the edges
are aligned with the primary x and y axes of the Cartesian coordinate
system. Movement of the field is restricted to x or y translations.

However, the Celestial coordinate system is defined by the Equatorial
coordinate system, with coordinates specified by Right Ascension (RA)
and Declination (Dec). Because this is a spherical coordinate system,
unit angle in RA is foreshortened, with the degree of foreshortening
defined in Expression \ref{RAshort}
\begin{equ}[!h]
  \begin{equation}
angle \in RA = {\frac{True Angle}{cos(Dec)}}
  \end{equation}
\caption{\label{RAshort}Right Ascension foreshortening with Declination}
\end{equ}

By using this conversion, it is possible to approximate to a high degree
of accuracy a Cartesian coordinate system using RA and Dec; with a small
FoV of East-West size \(R\) and North-South size \(S\) about a target located at
point \(RA_t\) and \(Dec_t\).  Expression \ref{Rprime} defines a corrected angular size in RA direction ($R^\prime$)

\begin{equ}[!h]
  \begin{equation}
R^\prime = {\frac{R}{cos(Dec_t)}}
  \end{equation}
\caption{\label{Rprime}Definition of a corrected angular size along the RA direction (R$^\prime$)}
\end{equ}

Given these terms, Expression \ref{FoVDef} defines the FoV.

\begin{equ}[!h]
\begin{equation}
\begin{split}
&RA_t - {\frac{R^\prime}{2}} \leq RA \leq RA_t + {\frac{R^\prime}{2}} \\
&Dec_t - {\frac{S}{2}} \leq Dec \leq Dec_t + {\frac{S}{2}}
\end{split}
\end{equation}
\caption{\label{FoVDef}Definition of a FoV of size R x S centred on a target at
(\(RA_t\) , \(Dec_t\))}
\end{equ}

This definition is accurate to approximately 1\% for a FoV of area 15$^\prime$
square outside celestial polar regions as shown in Expression \ref{polar}.

\begin{equ}[!h]
  \begin{equation}
\begin{split}
&Given: R, S = 15^\prime , \lvert Dec\rvert \leq 66.5 \degree\\
&{\frac{\lvert\frac{R}{cos(Dec-S)}-\frac{R}{cos(Dec+S)}\rvert}{\frac{R}{cos(Dec)}}}\leq 0.01
\end{split}
  \end{equation}
\caption{\label{polar}Evaluation of the accuracy of the R$^\prime$ for areas away from the celestial pole.}
\end{equ}

As expresed here, the formula does not consider RA ``loop
around'' from 359.99\textdegree  to 0.00\textdegree ; resulting, for
example, in the exclusion of 0.23\% of the SDSS catalogue. Planned
enhancements to The Locus Algorithm will resolve these shortcomings.

We can therefore define the locus about any star on the sky located at
\(RA_t\) and \(Dec_t\) as the values of Right Ascension and Declination
as defined in Equation 2.

\hypertarget{candidate-zone}{%
\subsection{Candidate Zone}\label{candidate-zone}}

A Candidate Zone is defined as a region centred on the target, equal to
four times the area of the Field of View, within which any
reference star can be included in a Field of View with the target and
can therefore be considered as a candidate reference star in identifying
the optimum pointing. Conversely, stars outside the candidate zone
cannot be included in a Field of View with the target and cannot
therefore be considered as candidates reference stars. Hence the
Candidate Zone is the maximum region of sky centred on the target from
which to choose candidate reference stars when identifying an optimum
pointing for a given target. For a target positioned at coordinates
\(RA_c\) and \(Dec_c\) the resulting Candidate Zone is defined by Expression \ref{CZdef}.
\begin{equ}[!h]
  \begin{equation}
\begin{split}
&RA_t - R^\prime \leq RA_r \leq RA_t + R^\prime \\
&Dec_t - S \leq Dec_r \leq Dec_t + S
\end{split}
  \end{equation}
\caption{\label{CZdef}Definition of a Candidate Zone of size 2R x 2S centred on a
target with coordinates (\(RA_t\), \(Dec_t\)), in which zone reference stars with coordinates (\(RA_r\), \(Dec_r\)) can be found.}
\end{equ}


%\begin{figure}
%\centering
%\includegraphics{fig2.png}
%\caption{Modified image taken from the SDSS ``Navigate'' tool.
%The image showing fields (in red) needed to form a mosaic from which a
%Candidate Zone (green) centred on the target can be defined.}
%\end{figure}

\hypertarget{identification-and-filtering-of-reference-stars}{%
\subsection{Identification and Filtering of Reference
Stars}\label{identification-and-filtering-of-reference-stars}}

For each target, a list of candidate reference stars in its Candidate
Zone is produced based on the following criteria:

\begin{itemize}
\tightlist
\item
  Position: the reference star must be in the Candidate Zone as defined in Expression \ref{CZdef}.
\item
  Magnitude: the magnitude of the reference star (\(mag_r\)) must be within a user-defined limit (\(\Delta mag\)) of the target's magnitude (\(mag_t\)) as shown in Expression \ref{maglim}.
\item
  Colour: the colour index (e.g. \(g-r\)) of the reference star (\(col_r\)) must match the colour of the target (\(col_t\)) to
  within a user-specified limit(\(\Delta col\)) as shown in Expression \ref{collim}.
\item
  Resolvability: the reference star must be resolvable, i.e.~no other
  star that would impact a brightness measurements within a
  user-specified resolution limit.
\end{itemize}

All stars in the Candidate Zone which pass these initial filters become
the list of candidate reference stars for which loci will be
identified.

\begin{equ}[!h]
  \begin{equation}
\begin{split}
&mag_t - \Delta mag \leq mag_c \leq mag_t + \Delta mag \\
\end{split}
  \end{equation}
\caption{\label{maglim}Definition of the limits of mag difference between the target and references.}
\end{equ}

\begin{equ}[!h]
  \begin{equation}
\begin{split}
&col_t - \Delta col \leq col_c \leq col_t + \Delta col \\
\end{split}
  \end{equation}
\caption{\label{collim}Definition of the limits of colour difference between the target and references.}
\end{equ}

\hypertarget{identifying-the-effective-locus-for-each-candidate-reference-star}{%
\subsection{Identifying the Effective Locus for each Candidate Reference
Star}\label{identifying-the-effective-locus-for-each-candidate-reference-star}}

The locus associated with each candidate reference star must be
identified based on Equation \ref{FoVDef}. For the purposes of identifying Points
of Intersection, only the side surrounding a given candidate reference
star closest to the target need be considered. Hence, we can define the
effective locus for such a candidate reference star as a single line of
constant RA and a single line of constant Dec nearest the target star
as shown in Figure \ref{effective_locus}.

\begin{figure}
\center{\includegraphics[width=0.6\textwidth]{fig3.png}}
\caption{\label{effective_locus}Each effective locus is defined by assigning a pair
of RA and Dec coordinates for a corner point and a pair of lines North
or South and East or West from the corner point. In this diagram, each
candidate reference star is assigned a colour, and the effective locus
that corresponds to it is drawn in the same colour. Copied from \citet{creaner2016thesis}}
\end{figure}

Specifically, the effective locus can be defined as a corner point of
the locus and two lines: one of constant RA and the other of constant
Dec emanating from the corner point. 

Using the Equatorial Coordinate System discussed in Section \ref{definition-of-coordinate-system-and-locus}, with
coordinates of the target specified by \((RA_t, Dec_t)\)
and coordinates of the candidate reference star defined by
\((RA_r, Dec_r)\) and a size of FoV of horizontal
length R and vertical length S, the coordinates of the corner point
\((RA_c, Dec_c)\) are defined as shown in Expression \ref{CPDef}.
\begin{equ}[!h]
  \begin{equation}
\begin{split}
&RA_t \leq RA_r \Rightarrow RA_c = RA_r- {\frac{R^\prime}{2}}\\
&RA_t > RA_r \Rightarrow RA_c = RA_r+ {\frac{R^\prime}{2}} \\
&Dec_t \leq Dec_r \Rightarrow Dec_c = Dec_r- {\frac{S}{2}}\\
&Dec_t > Dec_r \Rightarrow Dec_c = Dec_r + {\frac{S}{2}}
\end{split}
  \end{equation}
\caption{\label{CPDef}Definition of the corner point (\(RA_c\), \(Dec_c\)) of the effective locus for a FoV of size R x S for a candidate reference star at (\(RA_r\), \(Dec_r\)) and a target at (\(RA_t\), \(Dec_t\)) }
\end{equ}
The directions \(DirRA\) (the direction of the line of constant RA) and 
\(DirDec\) (the direction of the line of constant Dec) of the lines is
determined by the RA and Dec of the candidate
reference star relative to that of the target are given in Expression \ref{DirDef} and as described below.

\begin{itemize}
\tightlist
\item
  If the RA of the candidate is greater than the target, the line of
  constant Dec is drawn in the direction of increasing RA
\item
  If the RA of the candidate is less than the target, the line of
  constant Dec is drawn in the direction of decreasing RA
\item
  If the Dec of the candidate is greater than the target, the line of
  constant RA is drawn in the direction of increasing Dec
\item
  If the Dec of the candidate is less than the target, the line of
  constant RA is drawn in the direction of decreasing Dec.
\end{itemize}

\begin{equ}[!htb]
  \begin{equation}
\begin{split}
&RA_t \leq RA_r \Rightarrow DirDec = +ive\\
&RA_t > RA_r \Rightarrow DirDec = -ive \\
&Dec_t \leq Dec_r \Rightarrow DirRA = +ive\\
&Dec_t > Dec_r \Rightarrow DirRA =-ive
\end{split}
  \end{equation}
\caption{\label{DirDef}Definition the directions (\(DirRA\), \(DirDec\)) of the lines from the corner point of that define the effective locus for a FoV of size R x S for a candidate reference star at (\(RA_c\), \(Dec_c\)) and given a target at (\(RA_t\), \(Dec_t\)).  In current implementations, these values are encoded as a binary switch, with 1 representing increasing (\(+ive\)) direction and 0 representing decreasing (\(-ive\)) direction.}
\end{equ}



\hypertarget{identifying-and-scoring-points-of-intersection-and-identifying-the-pointing.}{%
\subsection{Identifying and Scoring Points of Intersection and
identifying the
pointing.}\label{identifying-and-scoring-points-of-intersection-and-identifying-the-pointing.}}

The points where lines from any two loci are identified. This involves
comparing the corner point RA and Dec and direction of lines for one
locus with the corner point RA and Dec and direction of lines for a
second locus. In total eight variable associated with each two loci are
checked:

\begin{itemize}
\tightlist
\item
  For Locus 1: \(RA_{c1}\), \(Dec_{c1}\), \(DirRA_1\), \(DirDec_1\)
\item
  For Locus 2: \(RA_{c2}\), \(Dec_{c2}\), \(DirRA_2\), \(DirDec_2\)
\end{itemize}

Using these parameters, a check as to whether an intersection between
the two loci occurs is achieved as follows:  

\begin{itemize}
\tightlist
\item
  A line of constant Dec in the positive RA direction from the corner
  point of locus 1 will intersect with a line of constant RA in the
  positive Dec direction from the corner point of locus 2 if locus 1 has
  a lower RA than locus 2 and locus 1 has a higher Dec than locus 2.
\item
  A line of constant RA in the positive Dec direction from the corner
  point of locus 1 will intersect with a line of constant Dec in the
  positive RA direction from the corner point of locus 2 if locus 1 has
  a lower Dec than locus 2 and locus 1 has a higher RA than locus 2.
\end{itemize}

\ldots{} and so on. By checking all such possible combinations, all
pairs of loci in the field which result in a Point of Intersection are
identified and their RA and Dec noted. The above cases are mathematically 
expressed in Expression \ref{PoICheck}.

\begin{equ}[!h]
  \begin{equation}
\begin{split}
&if: DirDec_1 = +ive, DirRA_2 = +ive\\
&and: RA_{r1} < RA_{r2}\\
&and: Dec_{r1} > Dec_{r2}\\
&PoI\: exists\: at\: RA_p = RA_{c2}, Dec_p = Dec_{c1}\\
\\
&if: DirRA_1 = +ive, DirDec_2 = +ive\\
&and: RA_{r1} > RA_{r2}\\
&and: Dec_{r1} < Dec_{r2}\\
&PoI\: exists\: at\: RA_p = RA_{c1}, Dec_p = Dec_{c2}\\
\ldots{}
\end{split}
  \end{equation}
\caption{\label{PoICheck}Definition of a PoI (\(RA_p\), \(Dec_p\)) given several sample cases.}
\end{equ}

Subsequent to identification, each Point of Intersection is then scored.
This is achieved as follows:

\begin{itemize}
\tightlist
\item
  The number of reference stars in the Field of View centred on the
  Point of Intersection is counted.
\item
  Each reference star is assigned a \(rating\) value between 0 and 1 based
  on its similarity in colour to the target.
\item
  The ratings from all counted reference stars in the Field of View are
  combined into one overall \(score\) for the field (Figure \ref{PoIscores}).
\item
  The Point of Intersection with the highest score becomes the pointing
  for the target (Figure \ref{final}).
\end{itemize}

\begin{figure}[!htb]
\center{\includegraphics[width=0.6\textwidth]{fig4.png}}
\caption{\label{PoIscores}Points of Intersection (PoI), and their associated
score. In this diagram each star has a rating of 1, hence the score
associated with each PoI is equal to the number of reference stars
within a FoV centred at that PoI.Copied from \citet{creaner2016thesis}}
\end{figure}

\begin{figure}[!htb]
\center{\includegraphics[width=0.6\textwidth]{fig5.png}}
\caption{\label{final}Locus Algorithm. Target: white star. Pointing \& FoV:
blue. Reference stars and their loci: Fully in the FoV: greens. On the
edge of the FoV: yellows. Outside FoV: reds.  Modified from \citet{creaner2016thesis}}
\end{figure}

Scenarios can arise which result in an inability to identify an optimum
pointing for a given target for example if there are no, or a maximum of
one reference stars in the candidate zone; and if no points of
intersection arise -- a scenario which can arise if two (or more)
reference fall in one quadrant of the candidate zone resulting in
concentric loci, or where reference stars are too far apart in different
quadrants of the candidate zone in order for their loci to intersect.
All four of these scenarios are considered in practical implementations
of the locus algorithm aimed at identifying the optimum pointings for a
set of targets in a catalogue or list of targets.

In summary, the Locus Algorithm successfully identifies the RA and Dec
coordinates of the optimum pointing for a given target, where optimum
means a field of view with the maximum number of reference stars which
are similar in magnitude and colour to the target.

\section{EUGENE'S SECTION}
\label{example-implementation-of-the-locus-algorithm}
\textbf{Eugene's reworked section goes here}

\section{Applications of the Algorithm}
\label{Applications}
The pointings generated in this algorithm are the optimum pointing for each target given the observational parameters and scoring system used. This is of use to any observer aiming to carry out differential photometry observations as it automates the selection of pointing and identification of reference stars. Two main use-cases are envisiaged for this system: targeted use (where an observer wishes to identify the optimum pointing for a pre-determined target or set of targets) and catalogue generation (where many targets are submitted to the system and a set of scores and pointings are generated for each). 

\subsection{Targeted Use}
This scenario considers an observer who wishes to perform differential photometry observations of a pre-determined target(s).  The observer may use the algorithm to identify the optimum pointing for their target(s).  As illustrated for the target above, this pointing may be offset from the target but will always include the target and the reference stars with the maximum combined rating.  Software to identify this optimum pointing and select the reference stars centred on that pointing is available online at \citet{githubrepo}.  A web interface to this software is planned. 

\subsection{Catalogue Generation}
By submitting many targets at once, the optimum pointing for each can be determined and scores calculated for each.  These are output together and can be collated into a catalogue as demonstrated in \citet{quasarpaper}. The catalogues can then be used by an observer to select targets suitable for their needs by filtering the catalogue. For example, by using the scores for each target, targets with a higher score (and thus a better set of reference stars) can be selected for observation over targets with worse scores.  The catalogue production software generates lists of targets, their pointings and the scores associated with those pointings.  Users who select targets from the catalogues can then follow up their choice by idenifying the set of reference stars with the SQL queries found at \citet{githubrepo}.

\section{Conclusions}
\label{Conclusions}

This paper presents the locus algorithm, a novel system for the identification of optimal pointings for differential photometry given a set of parameters of the planned observation provided by the user. The algorithm is presented in in two stages. In the first stage, the concept of the algorithm is laid out and the steps of the algorithm are defined. In the second stage, a fully-worked example is shown applying the algorithm to the star \textit{SDSS1237680117417115655} for a 10$^\prime$ FoV. The premise of the algorithm is that a locus can be defined about any target or reference star, upon which a Field of View (FoV) can be centred and include that object at the edge of the FoV. At the Points of Intersection (PoI) between these loci, the set of targets which can be included in a FoV changes if the FoV moves in any direction. Therefore, these PoI are the essential points to compare when
determining the maximum number and quality of reference stars which can be included in a FoV of a given size.

\subsection{Summary of Algorithm}
The algorithmic description can be summarised in the following nine steps:

\textbf{Identify the target}: The target is identified by its coordinates in the RA/Dec coordinate system.

\textbf{Provide observational parameters}: The algorithm requires a FoV size, magnitude and colour difference limits and a resolution parameter to be provided.

\textbf{Define a Candidate Zone}: identify all stars which could be included in a FoV with the target by translating the position of the FoV in accordance with Expression (ref to CZ), which defines the Candidate Zone (CZ).

\textbf{Filter Candidate References}: for each star in the CZ, apply the filtering criteria magnitude, colour and resolution (Expression (ref to filters) to identify the candidate reference stars

\textbf{Calculate Loci}: The effective locus around each candidate reference star (i.e. the path upon which the FoV may be centred and include the candidate and the target) is calculated as per Expressions (ref to expressions).

\textbf{Identify Points of Intersection}: The points where the effective loci for two candidate reference stars intersect with one another are identified by combining their coordinates as shown in Expression (ref to PoI expression).

\textbf{Calculate Rating}: for each candidate reference star, a rating is calculated to indicate how closely its colour matches that of the target.

\textbf{Calculate Score}: For each PoI, a score is calculated by combining the ratings for each candidate reference star which can be included in a FoV centred on that target.

\textbf{Output Optimum Pointing}: The PoI with the best score is then selected as the optimised pointing for that target

\subsection{Summary of Demonstration}
For demonstration purposes, this algorithm has been applied to \textit{SDSS1237680117417115655} (referred to as the target). The same nine steps are
applied as outlined below

\textbf{Identify the target}: The target is found at (${RA = 346.65\degree}$, ${Dec =-5.04\degree}$)

\textbf{Provide observational parameters}: The FoV for the demonstration is $10^\prime$ (0.1667\textdegree). The magnitude difference limit is $\pm$ 2.0 mag. The colour difference limit is $\pm$0.1 mag. The resolution selected is 11$^{\prime\prime}$ (0.003\textdegree)

\textbf{Define a Candidate Zone}: The CZ around the target is defined by (${346.4827\degree \leq RA \leq 346.8173\degree{}}$, ${-5.206\degree \leq Dec \leq -4.8726\degree}$) and contains 1345 objects

\textbf{Filter Candidate References}: A candidate reference must meet the following filtering criteria (${12.648 \leq r \leq 16.648}$), (${0.634 \leq g-r \leq 0.834}$) and (${0.149 \leq r-i \leq 0.349}$) and have no star within 0.003\textdegree of it. Applying the filtering criteria leaves 14 Candidate References

\textbf{Calculate Loci}: For each of the 14 candidate reference stars, a locus is calculated, for example for Star 10, the locus is defined by ${RA_c=346.6463\degree{}}$ ${Dec_c=5.0697\degree{}}$, ${DirRA = +ive}$ and ${DirDec = +ive}$.

\textbf{Identify Points of Intersection}: Given 14 candidate reference stars and their corresponding Loci, there are 364 possible combinations that could be a PoI, which are generated by combining the coordinates of the cornerpoints of the Loci in pairs and checking the directions of the lines from each to determine whether a PoI actually exists. For example, a PoI exists between the loci for Stars 5 and 10 at RA=346.6463\textdegree{}, Dec = -5.1153\textdegree{}.

\textbf{Calculate Rating}: for each of the candidate reference stars a rating is calculated based on the difference in colour between the target and the reference star, for example, the rating of Star 10 is 0.830.

\textbf{Calculate Score}: Score is calculated by identifying the candidate reference stars which can be included in an FoV centred on each PoI and summing their ratings. For example, for the PoI between stars 5 and 10 above, seven candidate reference stars can be included in the FoV and their ratings sum to 3.87.

\textbf{Output Optimum Pointing}: By comparing the scores for each PoI, the one with the highest score can be determined to be the pointing used in this example at at RA=346.6463\textdegree{}, Dec = -5.1153\textdegree{} and Score 3.87.


\section{Further developments}
A software system has been developed to implement this algorithm and is available from \citet{githubrepo}. A paper describing the software has been submitted at \citet{softwarepaper}. Scaling that software to allow for large-scale use required the use a Grid Computing solution, described at \citet{gridsystempaper}. By supplying a set of quasars from the 4 th Quasar Catalogue of the Sloan Digital Sky Survey (SDSS) as targets and the remainder of SDSS as potential reference stars, it was possible to generate a catalogue of optimised pointings for 26779 quasars as per \citet{quasarpaper,ZenodoQuasarCatalogue}. Using all of the stars in SDSS as targets allowed for the optimum pointing to be determined for all such stars. These pointings are of use, for example, in the search for extrasolar planets by the transit method, where high-precision differential photometry is required \citep{exopaper,ZenodoXOPCatalogue}.


\section*{Acknowledgements}
This publication has received funding from PLACEHOLDER.

This paper makes use of data from the Sloan Digital Sky Survey (SDSS).  Funding for the SDSS and SDSS-II has been provided by the Alfred P. Sloan Foundation, the Participating Institutions, the National Science Foundation, the U.S. Department of Energy, the National Aeronautics and Space Administration, the Japanese Monbukagakusho, the Max Planck Society, and the Higher Education Funding Council for England. The SDSS Web Site is http://www.sdss.org/.

The SDSS is managed by the Astrophysical Research Consortium for the Participating Institutions. The Participating Institutions are the American Museum of Natural History, Astrophysical Institute Potsdam, University of Basel, University of Cambridge, Case Western Reserve University, University of Chicago, Drexel University, Fermilab, the Institute for Advanced Study, the Japan Participation Group, Johns Hopkins University, the Joint Institute for Nuclear Astrophysics, the Kavli Institute for Particle Astrophysics and Cosmology, the Korean Scientist Group, the Chinese Academy of Sciences (LAMOST), Los Alamos National Laboratory, the Max-Planck-Institute for Astronomy (MPIA), the Max-Planck-Institute for Astrophysics (MPA), New Mexico State University, Ohio State University, University of Pittsburgh, University of Portsmouth, Princeton University, the United States Naval Observatory, and the University of Washington.


%% References with BibTeX database:

\bibliographystyle{elsarticle-num-names}
\bibliography{Locus_Whole_bibtex}


%\newpage
%
%\hypertarget{references}{%
%\subsection*{References}\label{references}}
%\addcontentsline{toc}{subsection}{References}
%
%\hypertarget{refs}{}
%\leavevmode\hypertarget{ref-Aguado2018}{}%
%Aguado, D. S., Romina Ahumada, Andres Almeida, Scott F. Anderson, Brett
%H. Andrews, Borja Anguiano, Erik Aquino Ortiz, et al. 2018. ``The
%Fifteenth Data Release of the Sloan Digital Sky Surveys: First Release
%of MaNGA Derived Quantities, Data Visualization Tools and Stellar
%Library,'' December. \url{https://doi.org/10.3847/1538-4365/aaf651}.
%
%\leavevmode\hypertarget{ref-Honeycutt1992}{}%
%Honeycutt, R. K. 1992. ``CCD ensemble photometry on an inhomogeneous set
%of exposures.'' \emph{Publications of the Astronomical Society of the
%Pacific} 104 (676). IOP Publishing: 435.
%\url{https://doi.org/10.1086/133015}.
%
%\leavevmode\hypertarget{ref-Howell2000}{}%
%Howell, Steve B., and Steve B. 2000. ``Handbook of CCD Astronomy.''
%\emph{Handbook of CCD Astronomy / Steve B. Howell. Cambridge, U.K. ; New
%York : Cambridge University Press, C2000. (Cambridge Observing Handbooks
%for Research Astronomers ; 2)}.
%\url{http://adsabs.harvard.edu/abs/2000hccd.book.....H}.
%
%\leavevmode\hypertarget{ref-Milone2011}{}%
%Milone, E. F., and Jan Willem Pel. 2011. ``The High Road to Astronomical
%Photometric Precision: Differential Photometry.'' In \emph{Astronomical
%Photometry: Past, Present, and Future, Edited by E.f. Milone and c.
%Sterken. Astrophysics and Space Science Library, Vol. 373. Berlin:
%Springer, 2011. ISBN 978-1-4419-8049-6, P. 33-68}, 373:33--68.
%\url{https://doi.org/10.1007/978-1-4419-8050-2_2}.
%
%\leavevmode\hypertarget{ref-YOUNG1991}{}%
%YOUNG, ANDREW T., RUSSELL M. GENET, LOUIS J. BOYD, WILLIAM J. BORUCKI,
%G. WESLEY LOCKWOOD, GREGORY W. HENRY, DOUGLAS S. HALL, et al. 1991.
%``PRECISE AUTOMATIC DIFFERENTIAL STELLAR PHOTOMETRY.'' Astronomical
%Society of the Pacific. \url{https://doi.org/10.2307/40679676}.

\end{document}


